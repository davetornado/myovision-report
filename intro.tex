\section{Background and Project Goal}
Myogenesis is the process that describes the formation, growth, development as well as regeneration of muscle tissues in the body. In detail, mononucleated, partially differentiated precursor cells, also called myoblasts, proliferate, migrate, differentiate, align, fuse to ultimately form multinucleated myotubes. In their quiescent state, myoblasts are called myosatellite cells and require mechanical stress for activation. In vivo, myotubes mature into myofibers that stretch along the entirety of skeletal muscles \cite{tajbakhsh2009, bentzinger, bollen}.

Several diseases such as sarcopenia, muscle dystrophy, obesity, or diabetes severely affect the physiological homeostasis and lead to a loss of skeletal muscle mass and function (muscle atrophy) \cite{argiles}. In contrast, physical activity does not just lead to an increase of muscle mass and function (muscle hypertrophy), but also reduces glucose levels and lipid accumulation. Treatment of diseases that lead to muscle atrophy often requires a multifactorial approach. Therefore, candidate compounds, genes, proteins or metabolites are initially analyzed in large-scale in vitro experiments utilizing myoblast cell lines such as murine C2C12 cells \cite{bajaj}.

Given the spatiotemporal nature of myogenesis, it is crucial to identify and quantify individual myoblasts and fused myotubes in microscopy images to evaluate the effect of each compound or biological molecule on myoblast fusion or atrophy/hypertrophy. Other than simply counting the number of each cell type, the creation and morphological quantification of individual myotube masks enables a thorough analysis of several other biologically relevant parameters such as the differentiation state/extend or myotube area and diameter in addition to interaction of different myoblasts. Unfortunately, the manual creation of individual cell masks in large-scale experiments is tremendously time consuming and might take months to finish. Therefore, this project intends to obtain such quantitative statistics without resorting to heuristics by using two different instance segmentation models for cell nuclei and myotubes.

These models need to overcome two hurdles. First and foremost, they must be able to differentiate between overlapping instances. Besides coincidence, these overlaps might be caused by myogenesis itself or due to the threedimensional nature of the cells caused by acutal overlapping myotubes. Secondly, it should be robust in its predictions. Many times, microscopy images require preprocessing that can create artifacts by amplifying noise or blurring small scale structures. Therefore, the instance segmentation should aim to be as independent of the preprocessing as possible.

Taken together, the goal is not only to speed up otherwise time-consuming analyses, but also to improve reproducibility and eliminate all human bias within the process of instance segmentation. Human bias can occur at various points. Many times, it is difficult to distinguish whether a continuously bright region is one object or several ones. Furthermore, the quality of preprocessing can influence the number of counted cells because some instances may be too dim to spot with the naked eye.

This work is structured in the following manner: in Sec.~\ref{secunsupervised} classical unsupervised computer vision methods are applied to our datasets. Their advantages and drawbacks are discussed qualitatively. In Sec.~\ref{secstardist} and Sec.~\ref{secsam} the theory behind \texttt{Stardist} and \texttt{SAM} respectively is touched upon. The former model is used to segment the myoblasts and the latter segments the myotubes. In Sec.~\ref{secdataset} the available data is discussed more thoroughly and the concrete implementation of how \texttt{SAM} was fine-tuned for our purposes is described. Finally, the performance of both models can be found in Sec.~\ref{secperformance}.
