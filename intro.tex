\section{Project Goal}
Myogenesis is the process describing the formation, growth, development, and regeneration of muscle tissue in the body. Therein mononucleated, partially differentiated precursor cells, also called myoblasts, proliferate, migrate, differentiate, and eventually fuse and thereby form multinucleated fibers known as myotubes. The myotubes then mature into myofibers. For this project, images of myotubes and myoblasts from muscle tissue obtained from screening experiments performed on mice were made available.

Given the spatiotemporal nature of myogenesis, it is instrumental to be able to segment both myotubes and myoblasts from microscopy images since it enables more systematic and quantitative research in this area. Other than simply counting the number of cells, creating an overlay of myotubes and myoblasts in order to find how many myoblasts are encapsulated by a myotube can tell you what stage of myogenesis certain cells are in. This also tells you how many cells are quiescent and yet to differentiate. In a similar vein, this holds true for the measurement of the area and diameter of myotubes. This project intends to obtain such quantitative statistics without using heuristics. To this end, two models are needed: one that performs instance segmentation for the myoblasts and one for the myotubes. 

These models need to overcome two hurdles. First and foremost, it must be able to differentiate between overlapping instances. Besides coincidence, these overlaps may stem from the myogenesis itself, e.g. during cell fusion. It may also be due to the fact that samples is not truly two dimensional such that cells from underneath can cause such overlaps by shining through. Secondly, it should be robust in its predictions. Many times, microscopy images require preprocessing which can create artifacts, amplifying noise, or blur small scale structures. The instance segmentation should aim to be as independent of the preprocessing as possible.

All in all, the goal is not only to speed up otherwise time consuming analyses, but also to improve reproducibility and eliminate all human bias within the process of segmentation. Human bias can occur at various points. Many times it is difficult to distinguish whether a continuosly bright region is one object or several ones in actuality. Furthermore, the quality of preprocessing can influence the number of counted cells because some instances may be too dim too spot with the naked eye.
