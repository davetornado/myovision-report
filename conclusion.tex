\section{Conclusion and Outlook}
In our exploration of models for segmenting myotube images and their corresponding cell nuclei, Stardist emerged as highly efficient with minimal need for modification. Its success can be attributed to the relatively homogeneous shapes of nuclei, which align well with methods that rely on rigid shape approximation. On the other hand, \texttt{SAM}, despite living up to its reputation as a foundational model through its impressive zero-shot performance, required further training to fully automate our segmentation approach. This necessity led to the development of \texttt{MyoSAM}. Our objective was to create a tool, facilitating both inference and labelling tasks, accessible to medical experts without the need for programming expertise. \texttt{MyoSAM} was integrated into our tool in versatile ways, supporting automatic segmentation via the auto mask generator and enabling interactive segmentation. Both functionalities significantly streamline the segmentation process of myotube images. Recognizing the scarcity of labelled data in this domain, we have made our dataset and training code publicly available, hoping to spur further advancements in the field of musculoskeletal research.

However, \texttt{MyoSAM}, while effective, is not without its limitations. The primary challenge lies in the scarcity and lack of diversity in our dataset, which comprised only 19 annotated images. These images do not adequately represent the wide variability in myotubes, characterised by their complex and heterogeneous shapes, sizes, and colours. Additionally, our training was constrained by limited computational resources, restricting our ability to conduct extensive experiments, such as hyperparameter optimization.
We encourage domain experts to use our annotation tool to gather more data, highlighting the critical need for \texttt{MyoSAM} to be trained on a more diverse and expansive dataset. While we considered the entire \texttt{SAM} model for fine-tuning, more targeted approaches, such as adapter tuning techniques like LoRA or focusing on specific network components, may offer more efficient pathways for improvement. Optimising the auto mask generator, which underpins the automatic segmentation functionality, could enhance replicability and segmentation quality, tailored to general or specific use cases. Moreover, training lightweight architectures like U-Net, utilising our tool for data generation, presents an alternative strategy worth exploring.
Despite these challenges, we are encouraged by the advancements \texttt{MyoSAM} represents over \texttt{SAM}, especially considering the limited size of our dataset. \texttt{MyoSAM} mitigates human error and inter-rater variability, paving the way for more consistent and replicable analyses. This project marks a promising beginning, with ample room for improvement through training on larger, more representative datasets. We are optimistic about the potential and current capabilities of our tool.
