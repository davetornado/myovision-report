\section{Segmentation using Unsupervised Methods}
In the beginning, established, unsupervised methods were used to segment the images in order to gain a better understanding of the data. Unsupervised learning includes algorithms which intend to find regularities, structures, or patterns within unlabelled datasets. As they require no ground truth labelled data and are not too computationally expensive, they are a great fit for exploratory analysis. The results of two classical methods (watershed and \textcolor{red}{tbd}) applied to the given images is discussed in the following.  
\subsection{Watershed}
Little math is necessary to understand how segmentation using watersheds functions. First, the image needs transformed to grayscale because the resulting single channel needs to be thought of as the a third dimension defining a height profile or topography. In case of a uint8 encoding, the height may take values between 0 and 255. Each pixel can be either of the three following types: a (regional) minimum, a catchment basin or watershed of that minimum, or watershed lines. The first type of pixel is self-explainatory. Continuing with the metaphor, a pixel of second or third type can be thought of in the following manner: picture the position and intensity of the pixel as defining the starting point on the 3d topography bdefined by the grayscale image. Placing a drop of water on this location can either have it run down (second type) or stay put (third type). All the points where water would run downhill are known as watersheds. All the other points that are not minima define crests, which are the divide (or watershed) lines, beyond which water would not move at all. Iterating over possible intensities starting from the lowest one in the image, or, analogously, flooding the 3d landscape by poking a hole in the minimum, defines connected areas, or collections of water within a basin, around every regional minimum. Continued flooding will have the water level rise until the first two connected areas merge into one. To prevent that a dam would need to be built whose locations defines the pixel of the watershed lines. How to properly construct such dams is thoroughly explained in \textcolor{red}{ref Gonzales + Couprie/Bertand;  overlap cannot be resolved} The resulting watershed lines are then interpreted as the boundaries of an instance.

Based on this intuition, two observations can be made. Firstly, on first glance a catchment basin can have the shape of a myotube or cell nuclei making watershed a sensible segmentation method. Secondly, this method requires the instances to have low grayscale values. This implies that images need to be processed before applying the watershed algorithm since cells are accumulations of high intensity areas. The most naive approach would be a simple inversion of the image. But this can lead to oversegmentation due to noisy sections. More sophisticated approaches either are based on image gradient or a distance transform applied to a binary representation of the original image. The latter approach is used in this report and will be concretized before long.

\textcolor{red}{from these theoretical considerations: cannot resolve overlaps;}


