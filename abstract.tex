\selectlanguage{english}
\begin{abstract}
	Myogenesis, the process of muscle tissue formation and regeneration, involves the proliferation, migration, differentiation, and fusion of so called myoblasts, which ultimately form multinucleated myotubes and mature into myofibers. To gain a better understanding of myogenesis, deep learning models are proposed for accurate instance identification to quantify key metrics of individual nuclei and myotube instances in microscopy images. This approach is aimed at automating the time-consuming manual annotation process, ensuring reproducibility and eliminating human bias, while maintaining a high segmentation performance. The task consists of identifying two main objects: the homogeneously shaped nuclei, for which the pre-trained \texttt{Stardist} model was used, and the myotubes, which possess a very heterogeneous structure, for which a foundation model had to be employed due to the limited resources available and the lack of open source material. The Segment Anything Model (\texttt{SAM}) developed by Meta AI and trained on over 11 million diverse images and 1 billion corresponding masks has proven to be a promising option for this purpose, and hence been utilised in various ways. Its zero-shot performance was used in a custom-built annotation tool to significantly speed up the data labelling process and later fine-tuned to improve its understanding of myotube images, resulting in a model called MyoSAM.
\end{abstract}
\selectlanguage{ngerman}
\begin{abstract}
	\textcolor{red}{german text}
\end{abstract}
\selectlanguage{english}