\selectlanguage{english}
\begin{abstract}
	Myogenesis, the process of muscle tissue formation and regeneration, involves the proliferation, migration, differentiation, and fusion of so called myoblasts, which ultimately form multinucleated myotubes and mature into myofibers. To gain a better understanding of myogenesis, deep learning models are proposed for accurate instance identification to quantify key metrics of individual nuclei and myotube instances in microscopy images. This approach is aimed at automating the time-consuming manual annotation process, ensuring reproducibility and eliminating human bias, while maintaining a high segmentation performance. The task consists of identifying two main objects: the homogeneously shaped nuclei, for which the pre-trained \texttt{Stardist} model was used, and the myotubes, which possess a very heterogeneous structure, for which a foundation model had to be employed due to the limited resources available and the lack of open source material. The Segment Anything Model (\texttt{SAM}) developed by Meta AI and trained on over 11 million diverse images and 1 billion corresponding masks has proven to be a promising option for this purpose, and hence been utilised in various ways. Its zero-shot performance was used in a custom-built annotation tool to significantly speed up the data labelling process and later fine-tuned to improve its understanding of myotube images, resulting in a model called MyoSAM.
\end{abstract}
\selectlanguage{ngerman}
\begin{abstract}
	\textcolor{red}{Die Myogenese, der Prozess der Bildung und Regeneration von Muskelgewebe, umfasst die Proliferation, Migration, Differenzierung und Fusion sogenannter Myoblasten, die letztendlich multinukleäre Myotuben bilden und sich zu Myofasern entwickeln. Um ein besseres Verständnis der Myogenese zu erlangen, werden Deep-Learning-Modelle vorgeschlagen, um eine genaue Instanzerkennung zur Quantifizierung wichtiger Metriken einzelner Zellkerne und Myotubeninstanzen in Mikroskopiebildern zu ermöglichen. Dieser Ansatz zielt darauf ab, den zeitaufwändigen manuellen Annotationsprozess zu automatisieren, Reproduzierbarkeit sicherzustellen und menschliche Vorurteile zu eliminieren, während eine hohe Segmentierungsleistung beibehalten wird. Die Aufgabe besteht darin, zwei Hauptobjekte zu identifizieren: die homogen geformten Zellkerne, für die das vortrainierte \texttt{Stardist}-Modell verwendet wurde, und die Myotuben, die eine sehr heterogene Struktur aufweisen, für die aufgrund der begrenzten Ressourcen und des Mangels an Open-Source-Material ein Grundlagenmodell eingesetzt werden musste. Das von Meta AI entwickelte und auf über 11 Millionen verschiedenen Bildern und 1 Milliarde entsprechenden Masken trainierte Segmentierungsalles-Modell (\texttt{SAM}) hat sich als vielversprechende Option für diesen Zweck erwiesen und wurde daher auf verschiedene Weise genutzt. Seine Zero-Shot-Leistung wurde in einem benutzerdefinierten Annotationswerkzeug verwendet, um den Prozess der Datenbeschriftung signifikant zu beschleunigen, und später feinabgestimmt, um sein Verständnis für Myotubenbilder zu verbessern, was zu einem Modell namens MyoSAM führte.}
\end{abstract}
\selectlanguage{english}